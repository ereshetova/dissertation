OS kernel hardening is big area, lis main vectors, we chose to work on two, explain why these two. 

Outline the common problems in OS kernel hardening: long lifecycle of bugs (even in upstream), lack of fix propagation to all devices, lack of knowledge, need to eliminate bug classes or exploitation methods, etc.

\section{Security of Berkeley Packet Filter}

Having a powerful mechanism embedded into OS kernel is always very attractive for attackers. 
Linux kernel has such mechanism: in-kernel virtual machine available for unprivileged execution from userspace. 
Its range of use has only been constantly extending and examples of past attacks and hardening mechanisms exist, especially on JIT part.

We showed that current hardening was not enough and it proved the case for a proper hardening solution to be finally merged into the kernel. Publication 5. 

\section{Kernel memory safety}
A lot of prior work covers memory safety issues, but no practical solution is used to date inside linux kernel itself (only debugging).
Present both solutions (publication 6).


