\begin{quoting}
Find a nice quote here.
\end{quoting}

\section{Motivation}

Nowadays platform security is an essential component of many devices around us: from smartphones and tablets to smart watches and glasses, from home entertainment and video surveillance systems to smart home appliances, from in-vehicle infotainment to flying drones. More and more functionality that used to be carried our manually by humans is automated with interconnected devices. This trend brings many additional challenges into standard platform security architectures that originally have been developed for solid and relatively closed systems, such as servers, military devices etc. Mobile devices, such as smartphones, were the first new wave of devices that required totally different approach for designing platform security due to their open nature, i.e. ability for a user to install third-party applications, as well as a wide range of users that it was targeting. A designer of platform security solution could not anymore assume that a user of a device is an experienced system administrator understanding a wide range of security risks and enforcing mechanisms. A mobile device user not only lacks this knowledge, but also in many case explicitly not willing to be involved in any of such matters or decisions: he or she just wants to accomplish a simple task of sending a message or installing a popular game. The situation gets even more severe as we start looking into various in-vehicle infotainment systems where user-driven open entertainment system might be partly co-allocated with car's instrument cluster devices with a very strict safety requirements and regulations. 

There is no such thing as the perfect platform security architecture, its design and implementation varies greatly based on targeted environment, use cases, device capabilities, cost factors and prioritized set of risks and threats. A lot of prior research both in academia and industry examines this problem from all different angles and aspects, but the practice shows that achieving a good balance is still hard. Many attacks are published regularly that find holes in or abuse different platform security mechanisms and lead to severe implications on user's privacy, overall safety, denial of service and others. The producers of various devices, OEMs, in turn are in a very hard position of delivering their solutions fast (due to a very competitive pace of modern industry) and at the same time satisfying security requirements of various involved stakeholders from ordinary users to various legislation. 

\section{Objectives}
\label{sec:Objectives}

The goal of this thesis is to look at different practical platform security challenges that modern mobile and embedded systems are facing today and will face in the nearest future and try to develop mechanisms and techniques that can help OEMs deliver more secure devices. While doing this we want to make sure that whenever a certain mechanism or solution is proposed, it satisfies the following requirements:

\begin{enumerate}
	\item Security.  
	\item Practical deployability.
	\item Usability.	
\end{enumerate}

While these aspects might sound obvious and self explanatory, the practical details can be more than challenging.   

\section{Outline}

This dissertation is based on the six original publications which are grouped into three main areas, all around various aspects of platform security. Each chapter starts with the relevant background and ends with a short discussion section that evaluates the contribution as well as lists open problems in the area. 

Chapter~\ref{sec:plat-sec} presents an overall view on the main aspects of any platform security solution and includes Publication I that is a survey of the platform security architectures for the most common mobile operating systems. While two out of four considered OSes, Symbian and MeeGo, have been terminated since, they played a big role in the development of mobile platform security architectures and methods. Chapter~\ref{sec:plat-sec} ends with a motivation for selecting the two specific focus areas for this dissertation, described in the next two chapters. 

Chapter~\ref{sec:ac-policies} takes a deep look into the process isolation and access control methods employed by modern operating systems and encompasses Publications II, III and IV. Publication II examines the relatively new and raising method for process isolation that is based on the OS-level virtualization and shows its shortcomings against the traditional mandatory access control schemes. Publication III takes a deep look on the mandatory access control mechanism (SEAndroid), employed by the most popular mobile operating system at the moment when OEMs had to go through the challenge of learning to create access control policies. The publication shows the challenges that developers are facing when trying to create the correct policy for their components, as well as most common pitfalls. It also presents a list of practical tools that can help OEMs, developers and security researchers to create better access control policies. Finally Publication IV presents one of such tools, SELint, that was considered the most helpful for this task. 

Chapter~\ref{sec:kernel-hardening} focuses on a very different challenge: a need to be able to have the core of an operating system, OS kernel, secured against the various attacks. This is a very important area, because an OS kernel is becoming more and more attractive target for the attackers due to both its high privileges and recent tightening of the userspace security. Publication V shows a powerful attack against a very commonly used Linux kernel subsystem, Just In Time (JIT) Compiler for Berkeley Packet Filter, which served as a motivator for a stricter hardening of this subsystem. In turn Publication VI proposes two methods for addressing the Linux Kernel OS memory safety and therefore preventing common Linux OS kernel vulnerabilities, such as various buffer overflows and use-after-frees. 

Finally, Chapter~\ref{sec:discussion} presents an overall evaluation of the methods proposed in this dissertation against the objectives set in Section~\ref{sec:Objectives} and attempts to give an outlook on the whole future mobile and platform security challenges. 

