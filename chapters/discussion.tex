
In this dissertation, we have presented the overall platform security model for mobile and embedded devices, as well as discussed some of its most challenging aspects, such as process and application isolation, access control and overall operating system (OS) kernel hardening.
The dissertation has also proposed a set of tools and mechanisms to help improve the above-mentioned aspects further: 
\begin{enumerate}
	\item the SEAL and SELint tools for SEAndroid mandatory access control (MAC) policy development and analysis (Chapter~\ref{sec:seandroid} and Publications II, III),
	\item the reference counter overflow protection mechanism for the mainline Linux kernel that prevents typical use-after-free vulnerabilities for shared kernel objects (Chapter~\ref{sec:kern-mem-ref-count} and Publication VI), and
	\item Memory Protection Extensions for Kernel (MPXK), a run-time technique to detect spatial memory errors in the mainline Linux kernel (Chapter~\ref{sec:kern-mem-out-of-bounds} and Publication VI). 
\end{enumerate}

While we argue that this dissertation's work presents a valuable contribution to the present-day mobile and embedded platform security research, it is also important to consider the future challenges and directions concerning the dissertation topics:

\begin{itemize}
	\item \textbf{Mobile platform security continues to evolve and further improve existing mechanisms.} While we likely won't see any big revolutionary changes in the mobile platform security architecture in the near future, the gradual step by step improvement of all its pieces will continue. The MAC mechanisms will remain a foundation for isolating processes and applications on mobile platforms, MAC policies will improve and become more fine-grained and precise. At the same time, the importance of the OS kernel hardening will prevail since more and more exploits are targeting the kernel itself~\cite{stoep2016android}. In addition, the interfaces towards security hardware will develop and extend in order to enable many new desired use cases in payments, transportation, healthcare, etc.
		
	\item \textbf{The security of embedded devices will become a major focus in both academia and industry.} Nowadays embedded devices are rather weak with regards to their security and are prone to many kind of attacks~\cite{Choo2016}. This is clearly unacceptable for the successful development of the Internet of Things (IoT) domain and must be addressed in the near future. If an embedded OS is based on the long-term supported mainline Linux kernel, it can greatly help in reducing the maintenance work, because the security fixes would be automatically backported by the community\footnote{Alternatively, a smaller footprint open-source collaborative OS, such as Zephyr~\cite{zephyr} or NuttX~\cite{NuttX}, can be used to achieve the same purpose.}. The OS kernel hardening effort is therefore of utmost importance here, especially because many embedded devices do not get software updates as often as mobile ones. Thus, their base kernel and (if present) userspace must be hardened to prevent the successful exploitation of developer's mistakes and bugs. The application and process isolation is not so important for embedded devices, because many of them nowadays do not have multi-process or multi-application environments, but rather have a monolithic execution environment. However, when more powerful embedded devices reach capabilities similar to mobile platforms (such as support for third-party application installation or different trust levels for processes), they will face similar platform security needs and challenges.
	
	\item \textbf{The decentralized aspects of IoT bring new security challenges.} While it is important to harden a single embedded device against various attacks, the real security challenge in IoT comes from protecting a distributed set of devices. Many IoT standards (such as OCF~\cite{ocf} or WoT~\cite{wot}) are being developed now that focus on interoperability of different devices and easy device discovery via standardized metadata. The goal is to enable as many different use cases as possible and make the IoT solutions that work easily out-of-the-box. However, it also brings additional security challenges that need to be taken into account~\cite{McCool2018}. Easy device discovery creates the potential for vulnerability scans and privacy leaks. Cross-device interoperability can decrease the overall security of the IoT network if devices support different security models. The semantic search on the available metadata brings the possibility for denial-of-service attacks and inferencing of sensitive information. Finally, IoT data traversal through many devices and potentially many security domains must be kept secure from one end point to another.
	
	\item \textbf{Various attacks against the central processing unit (CPU) and other hardware mechanisms are going to be on the rise.} The last decade saw a steep rise in numerous different attacks against CPU caches~\cite{lipp2016armageddon},~\cite{brasser2017software},~\cite{gras2017aslr},~\cite{irazoqui2017cache}, as well as other hardware parts, like dynamic random access memory (DRAM) (aka rowhammer~\cite{seaborn2015exploiting}). However, the recent Spectre~\cite{Kocher2018spectre} and Meltdown~\cite{Lipp2018meltdown} attacks have opened a totally new set of possibilities for performing cache-based side channel attacks using CPU capabilities such as branch prediction and speculative execution. The attacks work by tricking a victim into speculatively performing some operations that would not normally occur during program execution. As a result of this speculative execution, one can leak confidential information (such as various security credentials or simply the content of kernel memory) via a side channel (typically a cache) to the attacker. These attacks are very powerful, affect all CPU vendors to varying degrees, and are likely to stay and be developed further in the future since they affect one of the core parts of how modern CPUs work and meet their performance criteria. 			
% The first protective measure, Kernel Page-Table Isolation (KPTI)~\cite{kpti}, was merged into the mainline Linux kernel at the end of 2017 and aims to mitigate the Meltdown attack. It isolates kernel page tables from userspace and therefore removes the possibility to speculatively prefetch the kernel mapped data by a userspace process. While this measure does help in protecting against the Meltdown attack, it also exhibits a significant performance downgrade~\cite{kptiperf}. The other measure~\cite{variant1}, merged into the mainline Linux kernel, provides mitigation for the Spectre attack. It works by sanitizing the dereferencing of arrays in kernel code to remove the possibility of speculatively addressing a kernel array or a kernel pointer beyond its bounds. The downside of this measure is that in order to be effective, it needs to be used in each potentially vulnerable place in kernel source code.     

	
		\item \textbf{The increased focus on the security of the CPU and hardware creates the need for their formal verification.} As attackers discover new unexpected CPU or hardware behaviors and find a way to abuse them, security researchers and architects must develop techniques to verify that the underlying hardware parts behave as intended. This is especially important for some critical embedded systems, where a mistake can lead to severe safety consequences, such as high-risk industrial environments and the automotive sector. In this light, formal verification methods can be used more and more to obtain such guarantees. However, given the complexity of many modern hardware components (and especially CPUs) and their strict proprietary nature, this task is going to be particularly challenging. Additionally, many of these systems, even if formally verified, quite often have to interact with complex and rapidly changing external infrastructure. Such infrastructure cannot be formally verified and therefore the end behavior of the overall solution is hard to assess. Lastly, formal verification methods should not be considered an ultimate solution, because they are not able to catch the potential side effects in hardware, such as charge leaking in memory cells that makes attacks like rowhammer~\cite{seaborn2015exploiting} possible.	
		\end{itemize}
		
		
