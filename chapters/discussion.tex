\todo[inline]{This chapter needs writing}

Outlook for mobile and platform security: IoT, automation, machine learning (since we have a lot of data).

Outlook for mobile: Mandatory access control to strengthen, Great focus on kernel hardening since more exploits are coming towards kernel, strengthening the link to the security HW. 

Outlook for embedded: sticking with mainline Linux where possible (kernel hardening is important), but applications are developed on higher point of the stack: different higher level runtimes, standards, OCF, WoT, etc. Here problem moves away from separating process& apps (since most of device content is trusted and no 3rd party apps yet) to protecting a device itself and more even a distributed set of devices with the new set of challenges: easy device discovery brings a potential for vulnerability scans and privacy leaks, semantic search on available metadata brings DoS and inferencing of information, data traversing many security domains, etc.

And all of these devices handle enormous amount of data that also needs to be securely processed --> link to machine learning???


Do we need anything else than outlook here? What can we possibly conclude that apart that there is still enough work to be done everywhere? 

Mention smth on raising attacks on caches and indirect branch predictor as smth that platform security would have to take in to account as it goes. 

formal analysis? Critical embedded system that might be analyzed : make them simple, they have to work with complex systems that cannot be formally verified.

machine learning

mass market adaptivity - not suitable for fitting into crypto specified solutions. 


5-10 points for future outlook 


  