
In this thesis we have presented the overall platform security model for mobile and embedded devices, as well as discussed some of its nowadays most challenging aspects, such as process and application isolation, access control and the overall OS kernel hardening.
The thesis has also proposed a set of tools and mechanisms to help improving above mentioned aspects further: 
\begin{enumerate}
	\item the SEAL and SELint tools for SEAndroid MAC policy development and analysis,
	\item the reference counter overflow protection mechanism for the mainline Linux kernel that prevents typical use-after-free vulnerabilities for shared kernel objects,
	\item MPXK, a run-time technique to detect spatial memory errors in the mainline Linux kernel. 
\end{enumerate}

While we argue that this thesis's work presents a valuable contribution to the nowadays mobile and embedded platform security research, it is also important to consider the future challenges and directions concerning the thesis topics:

\begin{itemize}
	\item \textbf{The mobile platform security continues to evolve and further develop its existing mechanisms.} While we likely won't see any big revolutionary changes in the mobile platform security architectures, the gradual step by step improvement of all its pieces will continue. The mandatory access control (MAC) mechanisms will remain a foundation stone for isolating processes and applications on mobile platforms, MAC policies will improve and become more fine-grained and precise. At the same time the importance of the OS kernel hardening will prevail since more and more exploits are targeting the kernel itself~\cite{stoep2016android}. In addition, the interfaces towards security HW will develop and extend in order to enable many new desired use cases in payments, transportation, healthcare etc.  
	\item \textbf{The security of embedded devices will become a major focus in both academia research and industry.} Nowadays embedded devices are rather weak with regards to their security and are prone to many kind of attacks~\cite{Choo2016}. This is clearly unacceptable for the successful development of the IoT domain and must be addressed in the near future. If an embedded OS is based on the long-term supported mainline Linux kernel, it can greatly help reducing the maintenance work, because the security fixes would be automatically backported by the community\footnote{Alternatively a smaller footprint open-source collaborative OS, such as Zephyr~\cite{zephyr} or NuttX~\cite{NuttX}, can be used to achieve the same purpose.}. The OS kernel hardening effort is therefore of utmost importance here, especially because many embedded devices do not get software updates as often as mobile ones. Thus their base kernel and (if present) userspace must be hardened to prevent the successful exploitation of developer's mistakes and bugs. The application and process isolation is not so important for embedded devices, because many of them nowadays do not have multi-process or multi-application environments, but rather have a monolithic execution environment. However, when more powerful embedded devices reach similar capabilities like mobile ones (like having 3-rd party application installation or different trust levels for processes), they would face the similar platform security needs and challenges. 
	\item \textbf{The decentralized aspects of IoT bring new security challenges.} While it is important to harden a single embedded device against various attacks, the real security challenge in IoT comes from protecting a distributed set of devices. Many IoT standards (such as OCF~\cite{ocf} or WoT~\cite{wot}) are being developed now that focus on interoperability of different devices, easy device discovery via standardized metadata. The goal is to enable as many different use cases as possible and make the IoT solutions to work easily out-of-box. However, it all brings additional security challenges that need to be taken into account~\cite{McCool2018}. The easy device discovery brings potential for vulnerability scans and privacy leaks. The cross-device interoperability can decrease the overall security of the IoT network if devices support different security models. The semantic search on the available metadata brings a possibility for Denial-of-Service attacks and inferencing of sensitive information. And finally, the IoT data traversal through many devices and potentially many security domains must be kept secure from one end point to another.
	\item \textbf{Various attacks against CPU and other HW mechanisms are going to be on the raise.} Last decade saw a steep raise of numerous different attacks against CPU caches~\cite{lipp2016armageddon},~\cite{brasser2017software},~\cite{gras2017aslr},~\cite{irazoqui2017cache}, as well as other HW parts, like DRAM (aka rowhammer~\cite{seaborn2015exploiting}). However, the recent Spectre~\cite{Kocher2018spectre} and Meltdown~\cite{Lipp2018meltdown} attacks have opened a totally new set of possibilities for performing cache side channel attacks using such CPU capabilities as branch prediction and speculative execution. The core part of these attacks is to trick the victim into speculatively performing some operations that would not occur during the normal program operation flow and leak the victims confidential information (such as various security credentials or simply content of the kernel memory) via a side channel (typically cache-based) to the attacker. These attacks are very powerful, affect all CPU vendors to various degrees, and are likely to stay and being developed further in the future since they affect one of the core parts of how modern CPUs work and achieve their performance criteria. 		
		\item \textbf{The increased focus on the security of the CPU and HW parts bring a need for their formal verification.} At attackers discover new unexpected CPU or HW behaviors and find a way to abuse them, security researchers and architects must develop techniques to verify that the underlying HW parts behave as intended. This is specially important for some critical embedded systems, where a mistake can lead to severe safety consequences, such as high-risk industrial environments and automotive sector. In this light the formal verification methods can be used more and more to obtain such guarantees. However, given the complexity of many modern HW parts (and especially CPUs) and their strict proprietary nature, this task is going to be particularly challenging. Additionally many of these systems, even if formally verified, quite often have to interact with complex and fast pace changing external infrastructures. These infrastructures cannot be formally verified and therefore the end behavior of the overall solution is hard to assist.  	
		\end{itemize}
		
		
