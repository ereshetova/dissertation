
In this thesis we have presented the overall platform security model for mobile and embedded devices, as well as discussed some of its nowadays most challenging aspects, such as process and application isolation, access control and overall OS kernel hardening.
The thesis has also proposed the set of tools and mechanisms to help improving the above mentioned aspects further: 1) the SEAL and SELint tools for SEAndroid MAC policy development and analysis, 2) the reference counter overflow protection mechanism for the mainline Linux kernel that prevents typical use-after-free vulnerabilities for shared kernel objects, and 3) MPXK, a run-time technique to detect spatial memory errors in the mainline Linux kernel. 

While we argue that this thesis's work presents a valuable contribution for nowadays mobile and embedded platform security research, it is also important to consider the future challenges and directions for this domain:

\begin{itemize}
	\item \textbf{The mobile platform security continues to evolve and further develop its existing mechanisms.} The mandatory access control mechanisms will continue to be a foundation stone for isolating processes and applications, MAC policies will improve and become more fine-grained and precise. At the same time the big focus will continue to be on the OS kernel hardening since more and more exploits are targeting the kernel itself. In addition the interfaces towards security HW will develop further in order to enable many new use cases in payments, transportation, healthcare etc.  
	\item \textbf{The security of embedded devices becomes a major focus in both academia research and industry.} Nowadays embedded devices are rather weak with regards to security and prone to many kind of attacks\todo[inline]{provide references}. This is clearly unacceptable for the successful development of the IoT domain and must be addressed in the near future. Using the mainline Linux kernel whenever possible or other smaller footprint open source collaborative OSes (such as Zephyr \todo[inline]{list more OSes, provide references}) can help to maintain a fast pace moving OS with a high focus on security at the same time. The OS kernel hardening effort is also of utmost important here since many embedded device do not get updates as often as mobile ones, and therefore their base kernel and userspace (if present) must be hardened to prevent the successful exploitation of developer's mistakes and bugs. The application and process isolation is not so important for embedded devices, because many might not have any separate processes or applications at all, but rather have a monolithic one process execution structure. However, when more powerful embedded devices would reach similar capabilities like mobile ones (like having 3rd party application installation), they would face the similar platform security needs and challenges. 
	\item \textbf{The decentralized aspects of IoT bring new security challenges.} While it is important to harden a single embedded device against various attacks, the real security challenge in IoT comes from protecting a distributed set of devices. Many IoT standards\todo[inline]{provide references} are being developed now that focus on interoperability of different devices, easy device discovery via standardized metadata, etc. The goal is to enable as many different use cases as possible and make IoT solutions to work easily out-of-box. However, it all brings additional security challenges that has not been currently addressed in these standards\todo[inline]{provide reference to our paper in NDSS}. The easy device discovery brings a potential for vulnerability scans and privacy leaks. The cross-device interoperability can decrease the overall security of the IoT network if devices support different security models. The semantic search on the available metadata brings a possibility for Denial-of-Service attacks and inferencing of sensitive information. And finally IoT data traversing many devices and potentially many security domains must be kept secure from one end point to another.
	\item \textbf{Various attacks against CPU and other HW mechanisms are going to be on the raise.} Last decade saw a steep raise of numerous different attacks against CPU caches and overall cache synchronizing techniques. However, recent Spectrum and Meltdown attacks opened a totally new set of possibilities for performing the cache-inferring attacks using capabilities of branch and code predictors. These attacks are very powerful, affect all CPU vendors to various degrees, and likely to stay and being develop further for a while in the future since they affect one of the core part of how modern CPUs work and achieve their performance criteria. In addition other HW parts, like persistent memory, will continue to be attacker's focus. 		
		\item \textbf{The increased focus on the security of CPU and HW parts brings a need for their formal verification.} At attackers discover new unexpected CPU or HW behaviors and find a way to abuse them, security researchers and architects must develop techniques to verify that the underlying HW parts behave as intended. This is specially important for some critical embedded systems, where a mistake can lead to severe safety consequences, such as high-risk industrial environments, automotive etc. In this light the formal verification methods can be used more and more to obtain such guarantees. However, given the complexity of many modern HW parts (and especially CPUs) and their strict proprietary nature, this task is going to be particularly challenging. Additionally many of these systems, even if formally verified, quite often have to interact with complex and fast pace changing external infrastructures. These infrastructures cannot be formally verified and therefore the end behavior of the overall solution is hard to assist.  	
		\end{itemize}